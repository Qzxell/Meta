\documentclass{article}
\usepackage[utf8]{inputenc}
\usepackage{amsmath}
\usepackage{amsfonts}
\usepackage{amssymb}
\usepackage{graphicx}

\begin{document}

\section*{Descripción del Problema}

Se te da una secuencia $a = [a_1, a_2, \ldots, a_n]$ que consiste en $n$ enteros positivos.

Llamemos grupo a un conjunto de elementos consecutivos, llamado segmento. Cada segmento está caracterizado por dos índices: el índice de su extremo izquierdo y el índice de su extremo derecho. Denotamos por $a[l, r]$ a un segmento de la secuencia $a$ con el extremo izquierdo en $l$ y el extremo derecho en $r$, es decir, $a[l, r] = [a_l, a_{l+1}, \ldots, a_r]$.

Por ejemplo, si $a = [31, 4, 15, 92, 6, 5]$, entonces $a[2, 5] = [4, 15, 92, 6]$, $a[5, 5] = [6]$, $a[1, 6] = [31, 4, 15, 92, 6, 5]$ son segmentos.

Dividimos la secuencia dada $a$ en segmentos de manera que:

\begin{itemize}
    \item Cada elemento esté exactamente en un segmento.
    \item Las sumas de los elementos para todos los segmentos sean iguales.
\end{itemize}

Por ejemplo, si $a = [55, 45, 30, 30, 40, 100]$, entonces esta secuencia se puede dividir en tres segmentos: $a[1,2] = [55, 45]$, $a[3,5] = [30, 30, 40]$, $a[6,6] = [100]$. Cada elemento pertenece exactamente a un segmento, y la suma de los elementos de cada segmento es 100.

Definimos el grosor de la división como la longitud del segmento más largo. Por ejemplo, el grosor de la división del ejemplo anterior es 3.

Encuentra el grosor mínimo entre todas las divisiones posibles de la secuencia dada $a$ en segmentos de la manera requerida.

\section{2}

Recientemente, para su cumpleaños, Fedya recibió un arreglo $a$ de $n$ enteros dispuestos en un círculo. Para cada par de números vecinos ($a_1$ y $a_2$, $a_2$ y $a_3$, ..., $a_{n-1}$ y $a_n$, $a_n$ y $a_1$), la diferencia absoluta entre ellos es igual a 1.

Llamemos máximo local a un elemento que es mayor que ambos de sus elementos vecinos. También llamamos mínimo local a un elemento que es menor que ambos de sus elementos vecinos. Ten en cuenta que los elementos $a_1$ y $a_n$ son vecinos.

Desafortunadamente, Fedya perdió un arreglo, pero recordó en él la suma de los máximos locales $x$ y la suma de los mínimos locales $y$.

Dado $x$ e $y$, ayuda a Fedya a encontrar cualquier arreglo coincidente de longitud mínima.

\section{problem1}%
\label{sec:problem1}
Se te proporciona una cuadrícula de $n \times m$ de enteros no negativos $a$. El valor $a_{i,j}$ representa la profundidad del agua en la fila $i$ y columna $j$.

Un lago es un conjunto de celdas tal que:

\begin{itemize}
    \item Cada celda en el conjunto tiene $a_{i,j} > 0$.
    \item Existe un camino entre cualquier par de celdas en el lago yendo hacia arriba, abajo, izquierda o derecha un número de veces y sin pisar una celda con $a_{i,j} = 0$.
\end{itemize}

El volumen de un lago es la suma de las profundidades de todas las celdas en el lago.

Encuentra el mayor volumen de un lago en la cuadrícula.


\section{contestB}%
\label{sec:contestB}
Estás jugando un juego de computadora. El nivel actual de este juego se puede modelar como una línea recta. Tu personaje está en el punto 0 de esta línea. Hay $n$ monstruos tratando de matar a tu personaje; el $i$-ésimo monstruo tiene una salud igual a $a_i$ y está inicialmente en el punto $x_i$.

Cada segundo, sucede lo siguiente:

\begin{itemize}
    \item Primero, disparas hasta $k$ balas a los monstruos. Cada bala apunta a exactamente un monstruo y disminuye su salud en 1. Por cada bala, eliges su objetivo arbitrariamente (por ejemplo, puedes disparar todas las balas a un monstruo, disparar todas las balas a diferentes monstruos, o elegir cualquier otra combinación). Cualquier monstruo puede ser objetivo de una bala, independientemente de su posición y de otros factores;
    \item Luego, todos los monstruos vivos con salud 0 o menos mueren;
    \item Después, todos los monstruos vivos se mueven 1 punto más cerca de ti (los monstruos a la izquierda de ti aumentan sus coordenadas en 1, los monstruos a la derecha de ti disminuyen sus coordenadas en 1). Si algún monstruo llega a tu personaje (se mueve al punto 0), pierdes.
\end{itemize}

¿Puedes sobrevivir y matar a todos los $n$ monstruos sin permitir que ninguno de ellos llegue a tu personaje?


\section{3}%
\label{sec:3}

Un arreglo $a$ de longitud $m$ se considera bueno si existe un arreglo de enteros $b$ de longitud $m$ tal que se cumplen las siguientes condiciones:

\[
\sum_{i=1}^{m} a_i = \sum_{i=1}^{m} b_i
\]

\[
a_i \neq b_i \quad \text{para cada índice } i \text{ de } 1 \text{ a } m
\]

\[
b_i > 0 \quad \text{para cada índice } i \text{ de } 1 \text{ a } m
\]

Se te da un arreglo $c$ de longitud $n$. Cada elemento de este arreglo es mayor que 0.

Debes responder $q$ consultas. Durante la $i$-ésima consulta, debes determinar si el subarreglo $c_{l_i}, c_{l_i+1}, \ldots, c_{r_i}$ es bueno.


\section{section name}%
\label{sec:section name}

Se te da una secuencia $a_1, a_2, \ldots, a_n$. Puedes realizar operaciones en la secuencia. En cada operación, puedes elegir un entero $i$ $(1 \leq i < n)$ y intercambiar los elementos $a_i$ y $a_{i+1}$ de la secuencia, si $a_i + a_{i+1}$ es impar.

Determina si es posible ordenar la secuencia en orden no decreciente utilizando esta operación cualquier cantidad de veces.



\end{document}
